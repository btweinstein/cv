% CV.tex
%
% (c) 2002 Matthew Boedicker <mboedick@mboedick.org> (original author) http://mboedick.org
% (c) 2003-2007 David tJ. Grant <davidgrant-at-gmail.com> http://www.davidgrant.ca
% (c) 2007-2010 Todd C. Miller <Todd.Miller@courtesan.com> http://www.courtesan.com/todd
% (c) 2012-2014 Bryan T. Weinstein <btweinstein@gmail.com> http://people.seas.harvard.edu/~bweinstein/
%
% This work is licensed under the Creative Commons Attribution-ShareAlike 3.0 Unported License. To view a copy of this license, visit http://creativecommons.org/licenses/by-sa/3.0/ or send a letter to Creative Commons, 171 Second Street, Suite 300, San Francisco, California, 94105, USA.

\documentclass[letterpaper,11pt]{article}

%%%% Define various commands %%%%%%%

\newcommand{\emailAddress}{btweinstein@gmail.com}

%-----------------------------------------------------------
\usepackage[empty]{fullpage}
\usepackage{color}

\usepackage{hyperref}
\hypersetup{
colorlinks=true,
urlcolor=black,
pdftitle={Bryan T. Weinstein's CV},
pdfauthor={Bryan T. Weinstein},
pdfsubject={CV},
}

\usepackage{fancyhdr}
\usepackage[backend=biber,sorting=ynt,style=numeric,minbibnames=2]{biblatex}
%\usepackage{natbib}
%\bibliographystyle{unsrt}

\usepackage{enumitem}

\pagestyle{fancy}
\renewcommand{\headrulewidth}{0pt}
\renewcommand{\footrulewidth}{0pt}
\renewcommand{\footskip}{15pt}
\lfoot{\emph{\small{Bryan T. Weinstein}}}
\rfoot{\emph{\small{Fall 2017}}}
\fancyfootoffset{.2in}

\definecolor{mygrey}{gray}{0.80}
\raggedbottom
\raggedright

% Adjust margins to 0.5in on all sides
\addtolength{\oddsidemargin}{-0.5in}
\addtolength{\evensidemargin}{-0.5in}
\addtolength{\textwidth}{1.0in}
\addtolength{\topmargin}{-0.5in}
\addtolength{\textheight}{1.0in}


%-----------------------------------------------------------
%Custom commands
\newcommand{\resitem}[1]{\item #1 \vspace{-2pt}}
\newcommand{\skillitem}[1]{ \item{#1}  \vspace{-3pt}}
\newcommand{\skillheader}[1]{ \item \textbf{#1}  \vspace{-7pt}}


\newlist{myitemize}{itemize}{2}
\setlist[myitemize, 1]{leftmargin=*, labelindent=17pt, labelsep=11pt, label=\textbullet}
\setlist[myitemize, 2]{label=\normalfont\bfseries \textendash, labelsep=5pt, leftmargin=17pt}

\newcommand{\resheading}[1]
{
        \vspace{2pt}
        \phantomsection
        \addcontentsline{toc}{section}{#1}
        {\large \colorbox{mygrey}{
        \begin{minipage}{\textwidth}{
                \textbf{#1 \vphantom{p\^{E}}}}
        \end{minipage}}
        }
        \vspace{-11pt}
}

%I kind of cheat here to make the bullets go where I want.
\newcommand{\ressubheading}[4]{\vspace{-14pt}
\begin{tabular*}{7.0in}{l@{\extracolsep{\fill}}r}
                \\
                \textbf{#1} & #2 \\
                \textit{#3} & \textit{#4} \\
\end{tabular*}\vspace{-6pt}}

% Instead of defining a general command, I define a silly
% specific one

\newcommand{\educationsubheading}[8]{\vspace{-40pt}
\begin{tabular*}{7.0in}{l@{\extracolsep{\fill}}r}
                \\
                \\
                \\
                \textbf{#1} & #2 \\
                \textit{#3} & \textit{#4} \\
                \textit{#5} & \textit{#6} \\
                \textit{#7} & \textit{#8} \\
\end{tabular*}\vspace{-6pt}}

\newcommand{\resbelowsubheading}[2]{\vspace{-14pt}
\begin{tabular*}{7.0in}{l@{\extracolsep{\fill}}r}
                \\
                \textit{#1} & \textit{#2} \\
\end{tabular*}\vspace{-6pt}}

\newcommand{\award}[2]{\vspace{-5pt}
\begin{tabular*}{7.0in}{l@{\extracolsep{\fill}}r}
                \textbf{#1} & #2 
\end{tabular*}\vspace{-8pt}}

%\newlist{inlinelist}{itemize}{1}
%\setlist[inlinelist,1]{leftmargin=*,labelindent=60pt, labelindent=17pt}

% The inputs here are: Name (Dr. something), position, affiliation, relationship to me, web page, email, phone number

\newcommand{\reference}[6]{
\ressubheading{#1}{#2}{#3}{}
\vspace{-1pt}
\begin{itemize}
        \resitem{\emph{Relationship}: #4}
        \resitem{\emph{Email}: \href{mailto:#5}{#5}}
        \resitem{\emph{Phone}: #6}
\end{itemize}
}

%-----------------------------------------------------------

\begin{document}

%These are for my own use; they don't do anything.
\title{Bryan T. Weinstein's CV}
\author{Bryan T. Weinstein}
\date{Fall 2015}

\vspace{2pt}
\phantomsection
\addcontentsline{toc}{part}{Bryan T. Weinstein's CV}


\begin{center}
\textbf{\Large{Bryan T. Weinstein}}
\\
\vspace{3pt}
\large{\url{https://btweinstein.github.io/}}
\vspace{-5pt}
\end{center}

\begin{tabular*}{7.5in}{l@{\extracolsep{\fill}}r}
25 Dighton St, Apt. 1                   &       (585) 738-0690  \\
Brighton, MA 02135 &    \href{mailto:\emailAddress}{\emailAddress} \\
\end{tabular*}
\\

\vspace{0.1in}

%%%%%%%%%%%%%%%%%%%%%%%%%%%%%%%%%%%%%

\resheading{Education}

\begin{itemize}
\item \ressubheading{Harvard University}{Cambridge, MA}{PhD in Applied Physics}{Expected May 2018}     
\begin{itemize}
\resitem{Working Thesis Title: \textit{Microbial Evolutionary Dynamics and Transport}}
\resitem{Applied stochastic, random-walk methods (spatial stochastic differential equations) to model the evolutionary dynamics of growing microbial colonies}
\resitem{Utilized fluid and solid mechanics to simulate microbial colony morphology}

\end{itemize}


\item \ressubheading{Harvard University}{Cambridge, MA}{PhD Secondary Field: Computational Science and Engineering (CSE)}{Expected May 2018}
\begin{itemize}                   
\resitem{Completed advanced applied math and scientific computing courses}
\resitem{Learned state-of-the-art computational methods used in scientific research and data science}
\resitem{\textbf{Capstone:} Developed an OpenCL powered Lattice Boltzmann fluid mechanics simulation utilizing OpenGL for real-time visualization. }
\end{itemize}     
              
\item \ressubheading{Harvard University}{Cambridge, MA}{S.M. in Applied Physics}{November 2014}
        \begin{itemize}
        \resitem{Completed 12 courses: 4 physics core courses, 4 CSE courses, and 4 soft-matter/biophysics electives}
        \resitem{GPA: 3.95/4.00}
        \end{itemize}
\item
        \ressubheading{Case Western Reserve University}{Cleveland, OH}{Bachelor of Science in Engineering, Engineering Physics}{May 2012}
        \begin{itemize}
                \resitem{GPA: 4.00/4.00, Summa Cum Laude, Valedictorian }
                \resitem{Engineering Concentration: Aerospace Engineering}
                \resitem{Senior Project: Simulating Interactions between Confined Spins and Ferromagnetic Vortices}
        \end{itemize}
\end{itemize}

%%%%%%%%%%%%%%%%%%%%%%%%%%%%%%%%%%%%%%
\resheading{Certifications}

\begin{itemize}
\item \ressubheading{Engineer in Training (EIT)}{Ohio}{Active}{September 2012}
\begin{itemize}

\resitem{Successfully passed Fundamentals of Engineering Exam, the first step towards becoming a licensed Professional Engineer (PE)}
\end{itemize}

\end{itemize}

%%%%%%%%%%%%%%%%%%%%%%%%%%%%%%%%%%%%%%%%%%%%%%%%
\resheading{Engineering Skills}

\begin{myitemize}

\skillheader{Analytical and Numerical}
\begin{myitemize}
\skillitem{Experience creating experiments, models, and simulations to understand the transport of mass, momentum, and energy/heat in complex fluids}
\skillitem{Ability to mesh and simulate fluid flows containing chemical reactions using standard open-source tools (OpenFOAM) and custom-built GPU-powered Lattice Boltzmann tools}
\skillitem{Experience simulating multicomponent-multiphase flows using the Lattice Boltzmann technique}
\skillitem{Expert knowledge of Applied Mathematics, especially stochastic modeling involving the Master equation, the Fokker-Planck equation (PDEs), and (spatial) stochastic differential equations}
%\skillitem{Ability to efficiently create and calibrate physical models to experiment through core physics training}
\end{myitemize}

\clearpage

\skillheader{Experimental}
\begin{myitemize}
\resitem{Expert at designing and conducting biological and soft matter experiments involving complex fluids; four years of research in a molecular biology laboratory}
\resitem{Experienced using rheometers to quantify fluid rheology}
\resitem{Significant experience using microscopy to image microbes and computational tools to analyze the images}
\end{myitemize}

\end{myitemize}


%%%%%%%%%%%%%%%%%%%%%%%%%%%%%%%%%%%%%%%%%%%%%%%%
\resheading{Computational Skills}

\begin{myitemize}

\skillitem{Developed over 30 GitHub repositories and wrote hundreds of Jupyter/IPython notebooks to create scientific simulations and analyze experimental data  during my PhD (see my website above)}

\skillitem{Over 8 years of experience optimizing programs to run on multiple processors, graphics processing units (GPUs), and supercomputers}

\skillitem{Expert at using Jupyter/IPython Notebooks to explore, visualize, and analyze large tabular datasets and large collections of images}

\skillitem{Experienced at applying stochastic techniques to model and solve high-dimensional problems}

\skillheader{Languages for General Scientific Computing:}
\begin{myitemize}
\skillitem{Python, Cython, OpenCL, CUDA, C, C++, Java, Mathematica, Matlab}
\end{myitemize}

\skillheader{Selected Python Packages and Tools:}
\begin{myitemize}
\skillitem{IPython/Jupyter Notebook, matplotlib, seaborn, numpy, scipy, pandas, scikit-image, pymc3, cython, cython\_gsl, PyOpenCL, PyCuda}
\end{myitemize}

\skillheader{Fluid and Solid Mechanics Simulations:}
\begin{myitemize}
\skillitem{Lattice Boltzmann Method (custom-built code), OpenFOAM, SALOME, gmsh}
\end{myitemize}        

\skillheader{Image Analysis Tools}
\begin{myitemize}
\resitem{Python, OpenCL, ImageJ/Fiji}
\end{myitemize}
             
\end{myitemize}


%%%%%%%%%%%%%%%%%%%%%%%%%%%%%%%%%%%%%%

\resheading{Fellowships and Awards}

\begin{itemize}

\item \ressubheading{Institute for Applied Computational Science Scholarship}{Cambridge, MA}{Graduate Student}{September 2016 - September 2017}

\begin{itemize}
\resitem{Wrote proposal and won a \$25,000 student scholarship from Harvard's Institute for Applied Computational Science (IACS)}

\resitem{Used funds to further develop my IACS capstone: an OpenCL-powered Lattice Boltzmann fluid mechanics simulator utilizing OpenGL for real-time visualization}

\end{itemize}


\item \ressubheading{Department of Energy Office of Science Graduate Fellowship}{Washington, D.C.}{Graduate Student}{September 2012 - September 2015}

\begin{itemize}

\resitem{Wrote proposal to win a competitive fellowship that supports students pursuing training in areas relevant to Department of Energy (DOE). Selected out of 1,300 applicants; 50 fellowships awarded}
\resitem{Attended yearly conferences at National Laboratories; presented posters on my active research, networked with other DOE fellows and government officials}
                
\end{itemize}

\item \ressubheading{Harvard University Pierce Fellow}{Cambridge, MA}{Graduate Student}{September 2012 - September 2015}

\begin{itemize}
                
\resitem{Won fellowship awarded to the highest caliber PhD students accepted into Harvard's School of Engineering and Applied Sciences (SEAS). Selected out of 150 students; 8 fellowships awarded}
                                
\end{itemize}

\end{itemize}




%%%%%%%%%%%%%%%%%%%%%%%%%%%%%%%%%%%%%%
\resheading{Publications}

\renewcommand{\refname}{\vspace*{-12mm}}

\begin{refsection}[Me]
\nocite{*}
\printbibliography
\end{refsection}

%%%%%%%%%%%%%%%%%%%%%%%%%%%%%%%%%%%%%%%%%%%%%%%%

%\resheading{Conferences and Invited Presentations}

%\renewcommand{\refname}{\vspace*{-12mm}}

%\begin{refsection}[Conferences]
%\nocite{*}
%\printbibliography
%\end{refsection}




%%%%%%%%%%%%%%%%%%%%%%%%%%%%%%%%%%%%%%
\resheading{Professional Organizations}

\begin{itemize}
        \resitem{Tau Beta Pi Engineering Honor Society}
        \resitem{American Physical Society}
\end{itemize}






%%%%%%%%%%%%%%%%%%%%%%%%%%%%%%%%%%%%%%%%%%%%%%
\end{document}
