% CV.tex
%
% (c) 2002 Matthew Boedicker <mboedick@mboedick.org> (original author) http://mboedick.org
% (c) 2003-2007 David tJ. Grant <davidgrant-at-gmail.com> http://www.davidgrant.ca
% (c) 2007-2010 Todd C. Miller <Todd.Miller@courtesan.com> http://www.courtesan.com/todd
% (c) 2012-2014 Bryan T. Weinstein <btweinstein@gmail.com> http://people.seas.harvard.edu/~bweinstein/
%
% This work is licensed under the Creative Commons Attribution-ShareAlike 3.0 Unported License. To view a copy of this license, visit http://creativecommons.org/licenses/by-sa/3.0/ or send a letter to Creative Commons, 171 Second Street, Suite 300, San Francisco, California, 94105, USA.

\documentclass[letterpaper,11pt]{article}

%%%% Define various commands %%%%%%%

\newcommand{\emailAddress}{btweinstein@gmail.com}

%-----------------------------------------------------------
\usepackage[empty]{fullpage}
\usepackage{color}

\usepackage{hyperref}
\hypersetup{
colorlinks=true,
urlcolor=black,
pdftitle={Bryan T. Weinstein's Data Science Resume},
pdfauthor={Bryan T. Weinstein},
pdfsubject={CV},
}

\usepackage{fancyhdr}
\usepackage[backend=biber,sorting=ynt,style=numeric,minbibnames=2,language=american]{biblatex}
%\usepackage{natbib}
%\bibliographystyle{unsrt}

\usepackage{enumitem}

\pagestyle{fancy}
\renewcommand{\headrulewidth}{0pt}
\renewcommand{\footrulewidth}{0pt}
\renewcommand{\footskip}{15pt}
\lfoot{\emph{\small{Bryan T. Weinstein}}}
\rfoot{\emph{\small{Summer 2025}}}
\fancyfootoffset{.2in}

\definecolor{mygrey}{gray}{0.80}
\raggedbottom
\raggedright

% Adjust margins to 0.5in on all sides
\addtolength{\oddsidemargin}{-0.5in}
\addtolength{\evensidemargin}{-0.5in}
\addtolength{\textwidth}{1.0in}
\addtolength{\topmargin}{-0.5in}
\addtolength{\textheight}{1.0in}


%-----------------------------------------------------------
%Custom commands
\newcommand{\resitem}[1]{\item #1 \vspace{-2pt}}
\newcommand{\skillitem}[1]{ \item{#1}  \vspace{-3pt}}
\newcommand{\skillheader}[1]{ \item \textbf{#1}  \vspace{-7pt}}


\newlist{myitemize}{itemize}{2}
\setlist[myitemize, 1]{leftmargin=*, labelindent=17pt, labelsep=11pt, label=\textbullet}
\setlist[myitemize, 2]{label=\normalfont\bfseries \textendash, labelsep=5pt, leftmargin=17pt}

\newcommand{\resheading}[1]
{
        \vspace{2pt}
        \phantomsection
        \addcontentsline{toc}{section}{#1}
        {\large \colorbox{mygrey}{
        \begin{minipage}{\textwidth}{
                \textbf{#1 \vphantom{p\^{E}}}}
        \end{minipage}}
        }
        \vspace{-11pt}
}

%I kind of cheat here to make the bullets go where I want.
\newcommand{\ressubheading}[4]{\vspace{-14pt}
\begin{tabular*}{7.0in}{l@{\extracolsep{\fill}}r}
                \\
                \textbf{#1} & #2 \\
                \textit{#3} & \textit{#4} \\
\end{tabular*}\vspace{-6pt}}

%I define a new block for MITRE since I got promoted
\newcommand{\ressubheadingmitre}[6]{\vspace{-29pt}
\begin{tabular*}{7.0in}{l@{\extracolsep{\fill}}r}
				\vspace{15pt} % horrible hack to get things lined up
                \\
                \textbf{#1} & #2 \\
                \textit{#3} & \textit{#4} \\
                \textit{#5} & \textit{#6}
\end{tabular*}\vspace{-6pt}}

% Instead of defining a general command, I define a silly
% specific one

\newcommand{\educationsubheading}[8]{\vspace{-40pt}
\begin{tabular*}{7.0in}{l@{\extracolsep{\fill}}r}
                \\
                \\
                \\
                \textbf{#1} & #2 \\
                \textit{#3} & \textit{#4} \\
                \textit{#5} & \textit{#6} \\
                \textit{#7} & \textit{#8} \\
\end{tabular*}\vspace{-6pt}}

\newcommand{\resbelowsubheading}[2]{\vspace{-40pt}
\begin{tabular*}{7.0in}{l@{\extracolsep{\fill}}r}
                \\
                \textit{#1} & \textit{#2} \\
\end{tabular*}\vspace{-6pt}}

\newcommand{\award}[2]{\vspace{-5pt}
\begin{tabular*}{7.0in}{l@{\extracolsep{\fill}}r}
                \textbf{#1} & #2 
\end{tabular*}\vspace{-8pt}}

%\newlist{inlinelist}{itemize}{1}
%\setlist[inlinelist,1]{leftmargin=*,labelindent=60pt, labelindent=17pt}

% The inputs here are: Name (Dr. something), position, affiliation, relationship to me, web page, email, phone number

\newcommand{\reference}[6]{
\ressubheading{#1}{#2}{#3}{}
\vspace{-1pt}
\begin{itemize}
        \resitem{\emph{Relationship}: #4}
        \resitem{\emph{Email}: \href{mailto:#5}{#5}}
        \resitem{\emph{Phone}: #6}
\end{itemize}
}

%-----------------------------------------------------------

\begin{document}

%These are for my own use; they don't do anything.
\title{Bryan T. Weinstein's CV}
\author{Bryan T. Weinstein}
\date{Fall 2015}

\vspace{2pt}
\phantomsection
\addcontentsline{toc}{part}{Bryan T. Weinstein's Resume}


\begin{center}
\textbf{\Large{Bryan T. Weinstein}}
\vspace{-25pt}
\end{center}

\begin{tabular*}{7.5in}{l@{\extracolsep{\fill}}r}
9 Sigmund Way  &   (585) 738-0690  \\
Walpole, MA 02081 &    \href{mailto:\emailAddress}{\emailAddress} \\
\end{tabular*}
\\

\vspace{0.1in}


%%%%%%%%%%%%%%%%%%%%%%%%%%%%%%%%%%%%%


%%%%%%%%%%%%%%%%%%%%%%%%%%%%%%%%%%%%%

\resheading{Education}

\begin{itemize}
\item \ressubheading{Harvard University}{Cambridge, MA}{PhD in Applied Physics; Secondary Field in Computational Science and Engineering (CSE)}{May 2018}     
\begin{itemize}

\resitem{Mastered state-of-the-art computational methods used in scientific research and data science; completed advanced applied math and scientific computing courses} 

\begin{itemize}
\resitem{Wrote proposal and won a \$25,000 student scholarship from Harvard's Institute for Applied Computational Science (IACS)}
\resitem{Used funds develop an OpenCL (GPU) powered Lattice Boltzmann fluid mechanics simulator utilizing OpenGL for real-time visualization}
\end{itemize}

\resitem{Wrote two papers: verified my own experimental results with analytical mathematical models and simulations of probabilistic chemical reactions (spatial stochastic differential equations) coupled to transport (fluid flow, diffusion) using custom solvers}

\resitem{Won Pierce Fellowship and Department of Energy (DOE) Office of Science Graduate Fellowship: less than 5\% acceptance rate for each}

\resitem{GPA: 3.95/4.00}

\end{itemize}
              
\item
        \ressubheading{Case Western Reserve University}{Cleveland, OH}{Bachelor of Science in Engineering, Engineering Physics}{May 2012}
        \begin{itemize}
                \resitem{GPA: 4.00/4.00, Summa Cum Laude, Valedictorian. Aerospace Engineering Concentration}
        \end{itemize}
\end{itemize}


\resheading{Work Experience}

\begin{itemize}
\item \ressubheadingmitre{MITRE}{Bedford, MA}{Lead Modeling \& Simulation Engineer / Analyst}{April 2021 - Present}{Senior Modeling \& Simulation Engineer / Analyst}{August 2018 - April 2021}

\begin{itemize}
\item Rapidly developed innovative technical solutions to national security problems utilizing modeling, simulation, engineering, data science, and prototyping skills



\item Pioneered widespread usage of government physics and agent based probabilistic modeling tool (AFSIM) in conjunction with state-of-the-art computational methods to create system analyses
\begin{itemize}
\item Founded community of practice for AFSIM and associated tooling; now has 700+ members
\item Developed popular Git version-controlled repositories with CI/CD docker-based testing of agents and deployment for large team. Presented capabilities and results at national conferences
\item Utilized HPC to run many probabilistic simulations; analyzed results with Python, Jupyter

\end{itemize}

\item Led key parts of division's R\&D and work programs; utilized modeling and simulation to inform government decisions about dynamic control of (often autonomous) assets across domains to accomplish military objectives
\begin{itemize}
\item Mentored dozens of staff and led diverse teams of various sizes across classification levels to produce high quality and timely deliverables
\item Presented results to senior government stakeholders across the DOD and MITRE executive leadership to deliver maximum impact
\item{Proposed and procured over three million dollars in internal research funding to build and deploy a prototype (Django, Postgres, UI/UX) allowing humans to interact with our simulations to conduct wargames; used prototype to solve directly-funded government problems}
\end{itemize}

\item{Built custom simulations and analytical mathematical models to rapidly answer government questions when existing tools were insufficient}
\item{Won 8 awards celebrating impactful delivery of prototypes and analyses that helped inform decisions at the highest levels of the US government}
\end{itemize}
\end{itemize}

%%%%%%%%%%%%%%%%%%%%%%%%%%%%%%%%%%%%%%%%%%%%%%%%
\resheading{Computational \& Analytical Skills}

\begin{myitemize}

\skillitem{Over 13 years of experience optimizing programs to run on multiple processors, graphics processing units (GPUs), and supercomputers}

\skillitem{Expert at using Jupyter/IPython Notebooks to explore, visualize, and analyze large tabular datasets and large collections of images}

\skillitem{Experienced at applying stochastic techniques to model and solve high-dimensional problems}

\skillitem{Expert at rapidly creating new M\&S software tools to answer novel questions}

\skillitem{Ability to create and calibrate mathematical models to data through core physics training}
\skillitem{Expert knowledge of Applied Mathematics, especially stochastic modeling involving the Master equation, the Fokker-Planck equation (PDEs), and (spatial) stochastic differential equations}

\skillitem{Expert ability to create experiments, models, and numerical simulations to study the transport of mass, momentum, and energy coupled to probabilistic chemical reactions in complex fluids and materials}

\skillheader{Languages for General Scientific Computing:}
\begin{myitemize}
\skillitem{Python, Cython, OpenCL, CUDA, C, C++, Java, Mathematica, Matlab}
\end{myitemize}

\skillheader{Selected Python Packages and Tools:}
\begin{myitemize}
\skillitem{Jupyter Notebooks, matplotlib, seaborn, colorcet, numpy, scipy, pandas, pandera, scikit-image, pymc3, multiprocessing, Django, pytest, cython, cython\_gsl, mako, PyOpenCL, PyCUDA, poetry}
\end{myitemize}

\skillheader{Selected Software Development Tools:}
\begin{myitemize}
\skillitem{Docker, CI/CD, GitLab, Git, REST APIs, Flask, FastAPI, Pydantic, JIRA, Nexus Registries, VS Code, PyCharm, Vim}
\end{myitemize}

\skillheader{Fluid and Solid Mechanics Simulations:}
\begin{myitemize}
\skillitem{Lattice Boltzmann Method (custom-built code), OpenFOAM, SALOME, gmsh}
\end{myitemize}        

\skillheader{Image Analysis Tools}
\begin{myitemize}
\resitem{Python, OpenCL, ImageJ/Fiji}
\end{myitemize}

\skillheader{Selected Government Software}
\begin{myitemize}
\skillitem{AFSIM, pymission, SBSS, C2S, milsymbol, DIS}
\end{myitemize}
             
\end{myitemize}



%%%%%%%%%%%%%%%%%%%%%%%%%%%%%%%%%%%%%%
\resheading{Certifications}


\begin{itemize}
\item \ressubheading{Top Secret / SCI Clearance}{MITRE}{Active}{October 2020}
\item \ressubheading{Secret Clearance}{MITRE}{Active}{October 2019}
\item \ressubheading{Engineer in Training (EIT)}{Ohio}{Active}{September 2012}
\end{itemize}


%%%%%%%%%%%%%%%%%%%%%%%%%%%%%%%%%%%%%%
\resheading{Publications}

\renewcommand{\refname}{\vspace*{-12mm}}

\begin{refsection}[Me]
\nocite{*}
\printbibliography
\end{refsection}

%%%%%%%%%%%%%%%%%%%%%%%%%%%%%%%%%%%%%%%%%%%%%%%%

%\resheading{Conferences and Invited Presentations}

%\renewcommand{\refname}{\vspace*{-12mm}}

%\begin{refsection}[Conferences]
%\nocite{*}
%\printbibliography
%\end{refsection}




%%%%%%%%%%%%%%%%%%%%%%%%%%%%%%%%%%%%%%
%\resheading{Professional Organizations}
%
%\begin{itemize}
%        \resitem{Tau Beta Pi Engineering Honor Society}
 %       \resitem{American Physical Society}
%\end{itemize}






%%%%%%%%%%%%%%%%%%%%%%%%%%%%%%%%%%%%%%%%%%%%%%
\end{document}
